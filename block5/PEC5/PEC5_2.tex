\section*{2}

Determine la solución (máximo) del problema dual asociado al siguiente problema primal
\begin{equation*}
\begin{aligned}
    (P) \quad \min \quad & x_1 + 2 (x_2 - 1)^2 \\
    \text{sujecto a} \quad & - x_1 - x_2 + 1 \leq 0, \\
        & x_1, x_2 \in \R.
\end{aligned}
\end{equation*}

\noindent\rule{10cm}{0.4pt}

El problema dual asociado a $(P)$ se puede escribir como
\begin{equation*}
    \max_{u \geq 0} \inf_{(x_1, x_2) \in \R^2} x_1 + 2 (x_2 - 1)^2 - u (x_1 + x_2 - 1),
\end{equation*}
que es equivalente a
\begin{equation*}
    \max_{u \geq 0} \inf_{(x_1, x_2) \in \R^2}
        2 x_2^2
        - 4 x_2 
        + 2
        + u (1 - x_2)
        + (1 - u) (x_1),
\end{equation*}
vemos que si  $u < 1$ y $x_1 \rightarrow - \infty$ la function anterior tiende a $-\infty$,
por tanto $u \geq 1$,
pero si $u > 1$ entonces el ínfimo se alcanzaría cuando $x_1 \rightarrow \infty$ valiendo $-\infty$,
de modo que $u = 1$.
Y tenemos el problema
\begin{equation*}
    \min_{(x_1, x_2) \in \R^2}
        2 x_2^2
        - 5 x_2 
        + 3,
\end{equation*}
que tiene como mínimo $-\frac{1}{8}$,
y por tanto la solución del problema dual es $-\frac{1}{8}$.
