\section*{3}

Una persona dispone de 6000 euros para invertir.
Se le sugiere invertir en bonos de dos tipos A y B.
Los del tipo A tienen más riesgo y dan un interés anual del 10\%,
mientras que los del tipo B tienen menos riesgo, pero dan un interés anual del 7\%.
Esta persona decide invertir, como máximo, 3500 euros en bonos del tipo A,
y 2400 euros, como mínimo, en bonos del tipo B,
de forma que la cantidad invertida en bonos de tipo A no sea menor que la invertida en bonos de tipo B.
Su objetivo es obtener, con estas restricciones, el máximo interés.

Se pide formular el correspondiente problema de optimización.

\noindent\rule{10cm}{0.4pt}

Sean $e_A$ la cantidad de euros invertidos en bonos del tipo $A$ y $e_B$ la cantidad de euros invertidos en bonos del tipo $B$,
la función objetivo representa el interés anual que se va a recibir,
esta esta representada por $\frac{0.1 e_A + 0.07 e_B}{e_A + e_B}$,
suponiendo que ninguno de los dos bonos deja de pagar.

Por otro lado las restricciones que tenemos que tener en cuenta son no sobre pasar el capital total,
respetar el máximo y mínimo a invertir en los bonos del tipo $A$ y $B$,
y que la inversion en bonos del tipo $A$ no sea menor a la del tipo $B$.

Con todo esto tenemos el problema de optimización,
\begin{equation}
\begin{aligned}
    \text{Maximizar }   & \frac{0.1 e_A + 0.07 e_B}{e_A + e_B} \\
    \text{sujeto a }    & e_A + e_B \leq 6000 \\
                        & e_A \leq 3500 \\
                        & e_B \geq 2400 \\
                        & e_A \geq e_B \\
\end{aligned}
\end{equation}
