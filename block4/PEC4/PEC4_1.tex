\section*{1}

Dado el problema de optimización
\begin{equation*}
\begin{aligned}
    \min \quad & x_1 + x_2 \\
    \text{sujecto a} \quad & x_2 \leq x_1^3, \\
        & x_1 \in \R, \quad x_2 \geq 0.
\end{aligned}
\end{equation*}

\begin{enumerate}[label=(\alph*)]
    \item Demuestra que $\bar{x} = (0, 0)$ es una solución del problema de optimización.
    \item Se satisface MFCQ en el punto $\bar{x} = (0, 0)$?
    \item Se satisfacen las condiciones KKT en el punto $\bar{x} = (0, 0)$?
\end{enumerate}

\noindent\rule{10cm}{0.4pt}

\subsection*{(a)}

Dadas las restricciones
\begin{equation*}
    x_2 \leq x_1^3, \quad x_2 \geq 0,
\end{equation*}
tenemos que necesariamente
\begin{equation*}
    x_1 \geq 0,
\end{equation*}
como la funcion
\begin{equation*}
    f(x_1, x_2) = x_1 + x_2,
\end{equation*}
es positiva para todo $x_1, x_2 \geq 0$ tenemos que
\begin{equation*}
    f(0,0) = 0 \leq f(x_1, x_2), \; \forall (x_1, x_2) \in \R^{2}_{+},
\end{equation*}
por tanto $(0, 0)$ es un mínimo de la función en el conjunto factible y una solución del problema.


\subsection*{(b)}

Consideramos
\begin{equation*}
    g(x_1, x_2) =
    \begin{pmatrix}
        x_2 - x_1^3 \\
        -x_2 \\
    \end{pmatrix},
\end{equation*}
y el cono $C = \R^2_{+}$.
Entonces para todo $(y_1, y_2) \in \R^2$
\begin{equation*}
    g(0, 0) + g'(0, 0) ((0, 0) - (y_1, y_2))^T
    = (0, 0) + 
        \begin{pmatrix}
            0 & 1 \\
            0 & -1 \\
        \end{pmatrix}
        \begin{pmatrix}
            -y_1 \\
            -y_2 \\
        \end{pmatrix}
    = (-y_2, y_2),
\end{equation*}
claramente $(-y_2, y_2) \not\in - \text{int}(\R^2_{+})$,
por tanto no se cumplen las condiciones MFCQ.

\subsection*{(c)}

El sistema de condiciones KKT del problema viene dado por
\begin{equation*}
    \begin{pmatrix}
        1 \\
        1 \\
    \end{pmatrix} 
    + u_1
    \begin{pmatrix}
        -3 x_1^2 \\
        1 \\
    \end{pmatrix} 
    + u_2
    \begin{pmatrix}
        0 \\
        -1 \\
    \end{pmatrix} 
    =
    \begin{pmatrix}
        0 \\
        0 \\
    \end{pmatrix},
\end{equation*}
\begin{equation*}
    u_1(x_2 + x_1^3) = 0,
\end{equation*}
\begin{equation*}
    u_2(x_2) = 0,
\end{equation*}
\begin{equation*}
    u_1 \geq 0, \quad u_2 \geq 0.
\end{equation*}
Si $u_2 = 0$ implica que $1 + u_1 = 0$ y por tanto $u_1 = -1 < 0!$ contradiciendo la condición que $u_1 \geq 0$,
por otro lado si $u_1 = 0$ tenemos que $1 = 0!$ que es absurdo,
concluimos que ambas $g_1$ y $g_2$ están activas.

Pero esto hace que la asunción de regularidad del problema,
que existe $(y_1, y_2) \in \R^2$ tal que
\begin{equation*}
    \nabla g_i(0, 0)^T
    \begin{pmatrix}
        y_1 \\
        y_2 \\
    \end{pmatrix}
    < 0,
\end{equation*}
para todo $g_i$ activo.
Pero para $g_1$, tenemos
\begin{equation*}
    \nabla g_1(0, 0)^T
    \begin{pmatrix}
        y_1 \\
        y_2 \\
    \end{pmatrix}
    = (0, 1)
    \begin{pmatrix}
        y_1 \\
        y_2 \\
    \end{pmatrix}
    = y_2,
\end{equation*}
y para $g_2$, tenemos
\begin{equation*}
    \nabla g_2(0, 0)^T
    \begin{pmatrix}
        y_1 \\
        y_2 \\
    \end{pmatrix}
    = (0, -1)
    \begin{pmatrix}
        y_1 \\
        y_2 \\
    \end{pmatrix}
    = -y_2,
\end{equation*}
claramente no existe $y_2 \in \R$ tal que $y_2 < 0$ y $-y_2 < 0$,
por tanto el problema no cumple las condiciones KKT.