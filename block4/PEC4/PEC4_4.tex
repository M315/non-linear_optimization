\section*{4}

Encuentra $\bar{u} \in L^2_{\infty} [0, 1]$ un control optimo del problema
\begin{equation*}
\begin{aligned}
    \min \quad & \int^1_0 \left[ u_1(t) - \frac{1}{3}x_1(t) + 2 u_2(t) - \frac{2}{3} x_2(t) \right] dt \\
    \text{sujecto a} \\
        & \left.
        \begin{aligned}
            & \dot{x_1}(t) = 12 u_1(t) - 2 u_1(t)^2 - x_1(t) - u_2(t) \\
            & \dot{x_2}(t) = 12 u_2(t) - 2 u_2(t)^2 - x_2(t) - u_1(t) \\
        \end{aligned}
        \right\} \text{ a.e. on } [0,1], \\
        & x_1(0) = x_{0_1}, \quad x_2(0) = x_{0_2}, \\
        & \left.
        \begin{aligned}
            & u_1 \geq 0 \\
            & u_2 \geq 0 \\
        \end{aligned}
        \right\} \text{ a.e. on } [0,1], \\
\end{aligned}
\end{equation*}
donde $x_{0_1}, x_{0_2} \in \R$.

\noindent\rule{10cm}{0.4pt}

Para encontrar la solución del problema queremos usar el teorema 5.22 del libro de texto,
la ecuación adjunta del problema es
\begin{equation*}
    - \dot{p}(t) 
        = p(t)^{T} 
             \begin{pmatrix}
                 -1 & 0 \\
                 0 & -1 \\
             \end{pmatrix}
        + \frac{1}{3}
             \begin{pmatrix}
                 1 \\
                 2 \\
             \end{pmatrix} \text{ a.e en } [0, 1],
\end{equation*}
es decir que tenemos
\begin{equation*}
    - \dot{p}_1(t) = - p_1(t) + \frac{1}{3},
\end{equation*}
y
\begin{equation*}
    - \dot{p}_2(t) = - p_2(t) + \frac{2}{3},
\end{equation*}
por tanto
\begin{equation*}
    p_1(t) = c_1 e^t + \frac{1}{3}, \quad p_2(t) = c_2 e^t + \frac{2}{3}.
\end{equation*}

La condición de transversalidad viene dada por
\begin{equation*}
    - p(1)^T = (0, 0),
\end{equation*}
por tanto tenemos que
\begin{equation*}
\begin{aligned}
    p_1 (1) = c_1 e + \frac{1}{3} = 0 \Rightarrow c_1 = - \frac{1}{3} e^{-1}, \\
    p_2 (1) = c_2 e + \frac{2}{3} = 0 \Rightarrow c_2 = - \frac{2}{3} e^{-1}, \\
\end{aligned}
\end{equation*}
y resulta en que
\begin{equation*}
    p(t) = \frac{1}{3} (1 - e^{t - 1})
        \begin{pmatrix}
            1 \\
            2 \\
        \end{pmatrix}.
\end{equation*}

Ahora usamos el principio máximo de Pontryagin local para encontrar $\bar{u}$,
para ello nos ayudamos de la observación 5.20c del libro de texto,
y tenemos que
\begin{equation*}
    \frac{1}{3} (1 - e^{t - 1})
        \begin{pmatrix}
            1 \\
            2 \\
        \end{pmatrix}^T
        \begin{pmatrix}
            12 - 4 \bar{u}_1(t) & -1 \\
            -1 & 12 - 4 \bar{u}_2(t) \\
        \end{pmatrix}
    =
        \begin{pmatrix}
            1 \\
            2 \\
        \end{pmatrix},
\end{equation*}
por tanto
\begin{equation*}
    \frac{1}{3} (1 - e^{t - 1}) (10 - 4 \bar{u}_1 (t)) = 1
    \Rightarrow \bar{u}_1 (t) = \frac{5}{2} - \frac{3}{4 (1 - e^{t - 1})},
\end{equation*}
como $\bar{u}_1(t) \geq$ a.e. en $[0,1]$ y
\begin{equation*}
    \frac{5}{2} \leq \frac{3}{4 (1 - e^{t - 1})}
    \Leftrightarrow 1 - e^{t - 1} \leq \frac{3}{10}
    \Leftrightarrow \frac{7}{10} \leq e^{t-1}
    \Leftrightarrow 1 + \ln \left(\frac{7}{10}\right) \leq t,
\end{equation*}
por tanto
\begin{equation*}
    \bar{u}_1 (t) = \left\{
    \begin{aligned}
        & \frac{5}{2} - \frac{3}{4 (1 - e^{t - 1})} & \text{ a.e. en } [0, 1 + \ln(7/10)], \\
        & 0  & \text{ a.e. en } [1 + \ln(7/10), 1].
    \end{aligned}
    \right.
\end{equation*}

Y por otro lado
\begin{equation*}
    \frac{1}{3} (1 - e^{t - 1}) (23 - 8 \bar{u}_2 (t)) = 2
    \Rightarrow \bar{u}_2 (t) = \frac{23}{8} - \frac{3}{4 (1 - e^{t - 1})},
\end{equation*}
como $\bar{u}_2(t) \geq$ a.e. en $[0,1]$ y
\begin{equation*}
    \frac{23}{8} \leq \frac{3}{4 (1 - e^{t - 1})}
    \Leftrightarrow 1 - e^{t - 1} \leq \frac{6}{23}
    \Leftrightarrow \frac{17}{23} \leq e^{t-1}
    \Leftrightarrow 1 + \ln \left(\frac{17}{23}\right) \leq t,
\end{equation*}
por tanto
\begin{equation*}
    \bar{u}_2 (t) = \left\{
    \begin{aligned}
        & \frac{23}{8} - \frac{3}{4 (1 - e^{t - 1})} & \text{ a.e. en } [0, 1 + \ln(17/23)], \\
        & 0  & \text{ a.e. en } [1 + \ln(17/23), 1].
    \end{aligned}
    \right.
\end{equation*}

