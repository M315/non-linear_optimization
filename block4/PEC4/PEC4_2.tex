\section*{2}

Encuentra una solución optima para los siguientes problemas de optimización
\begin{enumerate}[label=(\alph*)]
    \item 
        \begin{equation*}
        \begin{aligned}
            \min \quad & (x - 3)^2 + (y - 2)^2 \\
            \text{sujecto a} \quad & x^2 + y^2 \leq 5, \\
                & x + y \leq 3, \\
                & x \geq 0, \quad y \geq 0.
        \end{aligned}
        \end{equation*}

    \item 
        \begin{equation*}
        \begin{aligned}
            \min \quad & (x - \frac{9}{4})^2 + (y - 2)^2 \\
            \text{sujecto a} \quad & x^2 - y \leq 0, \\
                & x + y - 6 \leq 0, \\
                & x \geq 0, \quad y \geq 0.
        \end{aligned}
        \end{equation*}

    \item 
        \begin{equation*}
        \begin{aligned}
            \max \quad & 3 x - y - 4 z^2 \\
            \text{sujecto a} \quad & x + y + z \leq 0, \\
                & -x + 2y + z^2 = 0, \\
                & x, y, z \in \R.
        \end{aligned}
        \end{equation*}

\end{enumerate}

\noindent\rule{10cm}{0.4pt}

\subsection*{(a)}

Las condiciones KKT para cualquier $(x, y) \in \R^2$ en este problema son
\begin{equation*}
    \begin{pmatrix}
        2x - 6 \\
        2y - 4 \\
    \end{pmatrix}
    +
    u_1
    \begin{pmatrix}
        2x \\
        2y \\
    \end{pmatrix}
    + u_2
    \begin{pmatrix}
        1 \\
        1 \\
    \end{pmatrix}
    + u_3
    \begin{pmatrix}
        -1 \\
        0 \\
    \end{pmatrix}
    + u_4
    \begin{pmatrix}
        0 \\
        -1 \\
    \end{pmatrix}
    =
    \begin{pmatrix}
        0 \\
        0 \\
    \end{pmatrix},
\end{equation*}
\begin{equation*}
    u_1(x^2 + y^2 - 5) = 0,
\end{equation*}
\begin{equation*}
    u_2(x + y - 3) = 0,
\end{equation*}
\begin{equation*}
    u_2(x + y - 3) = 0,
\end{equation*}
\begin{equation*}
    u_3 x = 0,
\end{equation*}
\begin{equation*}
    u_4 y = 0.
\end{equation*}

Consideramos el caso donde $g_1$ y $g_2$ están activas, por tanto $u_3 = u_4 = 0$, $x > 0$, $y > 0$, $u_1 \geq 0$, $u_2 \geq 0$,
y las condiciones
\begin{equation*}
\begin{aligned}
    (2x) (1 + u_1) + u_2 - 6 & = 0, \\
    (2y) (1 + u_1) + u_2 - 4 & = 0, \\
    x^2 + y^2 - 5 & = 0, \\
    x + y - 3 & = 0, \\
\end{aligned}
\end{equation*}
de la ultima igualdad tenemos que $x = 3 - y$,
substituyendo en la penúltima ecuación
\begin{equation*}
    9 - 6 y + 2 y^2 - 5 = 0 \Rightarrow y_1 = 1, \; y_2 = 2,
\end{equation*}
por tanto $x_1 = 2$, $x_2 = 1$.
Amabas soluciones cumplen $x > 0$ e $y > 0$,
veamos que cumplen las dos primeras igualdades,
para $(2, 1)$ tenemos que
\begin{equation*}
    \left\{
    \begin{aligned}
        4 u_1 + u_2 - 2 & = 0 \\
        2 u_1 + u_2 - 2 & = 0 \\
    \end{aligned}
    \right.
\end{equation*}
por tanto $u_1 = 0 \geq 0$ y $u_2 = 2 \geq 0$,
cumpliendo las condiciones KKT,
para $(1, 2)$ se tiene
\begin{equation*}
    \left\{
    \begin{aligned}
        2 u_1 + u_2 - 4 & = 0 \\
        4 u_1 + u_2 & = 0 \\
    \end{aligned}
    \right.
\end{equation*}
por tanto $u_1 = - 2$ y $u_2 = 8$, pero $u_1 < 0$ incumpliendo $u_1 \geq 0$,
por eso $(1, 2)$ no cumple las condiciones KKT.

Para ver que $(2, 1)$ es solución del problema queremos usar el Teorema 5.17 del texto base,
para ello es suficiente ver que $f, g_1, g_2$ son convexas.

La Hessiana de la función objetivo $f$ es
\begin{equation*}
\begin{pmatrix}
    2 & 0 \\
    0 & 2 \\
\end{pmatrix},
\end{equation*}
es semi-definida positiva y la Hessiana de $g_1$
\begin{equation*}
\begin{pmatrix}
    2 & 0 \\
    0 & 2 \\
\end{pmatrix},
\end{equation*}
es semi-definida positiva y la función $g_2$ es afina y por tanto convexa.
De modo que $f, g_1, g_2$ son convexas y por el teorema 5.17 del libro de texto $(2, 1)$ es solución del problema de optimización.


\subsection*{(b)}

Las condiciones KKT para cualquier $(x, y) \in \R^2$ son
\begin{equation*}
    \begin{pmatrix}
        2x - \frac{9}{2} \\
        2y - 4 \\
    \end{pmatrix}
    +
    u_1
    \begin{pmatrix}
        2x \\
        -1 \\
    \end{pmatrix}
    + u_2
    \begin{pmatrix}
        1 \\
        1 \\
    \end{pmatrix}
    + u_3
    \begin{pmatrix}
        -1 \\
        0 \\
    \end{pmatrix}
    + u_4
    \begin{pmatrix}
        0 \\
        -1 \\
    \end{pmatrix}
    =
    \begin{pmatrix}
        0 \\
        0 \\
    \end{pmatrix},
\end{equation*}
\begin{equation*}
    u_1 (x^2 - y) = 0,
\end{equation*}
\begin{equation*}
    u_2 (x + y - 6) = 0,
\end{equation*}
\begin{equation*}
    u_3 x = 0,
\end{equation*}
\begin{equation*}
    u_4 y = 0.
\end{equation*}

Consideramos el caso donde solo $g_1$ esta activo,
por tanto $u_2 = u_3 = u_4 = 0$, $x > 0$, $y > 0$, $u_1 \geq 0$ y $x + y - < 6$,
y las condiciones
\begin{equation*}
\begin{aligned}
    x^2 - y = 0, \\
    2x - \frac{9}{2} + 2 u_1 x = 0, \\ 
    2y - 4 - u_1 = 0, \\ 
\end{aligned}
\end{equation*}
de la primera condición tenemos que $x^2 = y$,
substituyendo en la ultima ecuación tenemos,
\begin{equation*}
    u_1 = 2x^2 - 4,
\end{equation*}
finalmente substituyendo en la segunda ecuación obtenemos
\begin{equation*}
    4 x^3 - 6x - \frac{9}{2} = 0 \Rightarrow x = \frac{3}{2},
\end{equation*}
por tanto $x = 1.5$, $y = 2.25$ y $u_1 = 0.5$ son solución del sistema,
y $(1.5, 2.25)$ satisfacen las condiciones KKT.

La Hessiana de la función objetivo es
\begin{equation*}
\begin{pmatrix}
    2 & 0 \\
    0 & 2 \\
\end{pmatrix},
\end{equation*}
es semi-definida positiva y la Hessiana de $g_1$
\begin{equation*}
\begin{pmatrix}
    2 & 0 \\
    0 & 0 \\
\end{pmatrix},
\end{equation*}
también es semi-definida positiva,
por tanto ambas son funciones convexas y podemos usar el Teorema 5.17 del libro de texto para concluir que $(1.5, 2.25)$ es un mínimo del problema de optimización.

\subsection*{(c)}

En este caso el problema es de maximization,
por tanto para aplicar el mismo método que en los apartados anteriores debemos reescribir el problema como
\begin{equation} \label{orginal_problem_c}
\begin{aligned}
    \min \quad & -3 x + y + 4 z^2 \\
    \text{sujecto a} \quad & x + y + z \leq 0, \\
        & -x + 2y + z^2 = 0, \\
        & x, y, z \in \R.
\end{aligned}
\end{equation}
En este caso las condiciones KKT para cualquier $(x, y, z) \in \R^3$ son
\begin{equation*}
    \begin{pmatrix}
        -3 \\
        1 \\
        8z \\
    \end{pmatrix}
    + u_1
    \begin{pmatrix}
        1 \\
        1 \\
        1 \\
    \end{pmatrix}
    + v_1
    \begin{pmatrix}
        -1 \\
        2 \\
        2z \\
    \end{pmatrix}
    =
    \begin{pmatrix}
        0 \\
        0 \\
        0 \\
    \end{pmatrix},
\end{equation*}
reescribimos las condiciones como
\begin{equation*}
\left\{
\begin{aligned}
    - 3 + u_1 - v_1 & = 0 \\
    1 + u_1 - 2 v_1 & = 0 \\
    8 z + u_1 + 2 z v_1 & = 0 \\
\end{aligned}
\right.
\end{equation*}
y también deben cumplir
\begin{equation*}
    u_1 (x + y + z) = 0,
\end{equation*}
\begin{equation*}
    -x + 2 y + z^2 = 0.
\end{equation*}

Usando la primera igualdad tenemos que
\begin{equation*}
    u_1 = 3 + v_1,
\end{equation*}
substituyendo en la segunda
\begin{equation*}
    4 + 3 v_1 = 0 \Rightarrow v_1 = \frac{4}{3},
\end{equation*}
y por tanto $u_1 = \frac{5}{3}$,
substituyendo ahora en la tercera igualdad obtenemos
\begin{equation*}
    8 z + \frac{5}{3} - \frac{8}{3} z = 0 \Rightarrow z = - \frac{5}{16},
\end{equation*}
ahora usando la penúltima igualdad
\begin{equation*}
    x = \frac{5}{16} - y,
\end{equation*}
y substituyendo en la ultima igualdad tenemos que
\begin{equation*}
    3 y - \frac{5}{16} + \frac{25}{256} = 0 \Rightarrow y = \frac{55}{768},
\end{equation*}
y por tanto $x = \frac{185}{768}$.

De modo que $(\frac{185}{768}, \frac{55}{768}, -\frac{5}{16})$ satisfacen las condiciones KKT.
Veamos que se cumplen las hipótesis del Teorema 5.17 del libro de texto.

La Hessiana de la función objetivo es
\begin{equation*}
    \begin{pmatrix}
        0 & 0 & 0 \\
        0 & 0 & 0 \\
        0 & 0 & 8 \\
    \end{pmatrix},
\end{equation*}
que es semi-definida positiva y por tanto convexa,
la función $g_1$ es linear y por tanto convexa.

Por otro lado la función $h_1$ no es quasi-linear, y este es un requisito del Teorema 5.17.
Consideramos la siguiente variación del problema $(\ref{orginal_problem_c})$ relajando la condición $h_1$,
\begin{equation} \label{new_problem_c}
\begin{aligned}
    \max \quad & 3 x - y - 4 z^2 \\
    \text{sujecto a} \quad & x + y + z \leq 0, \\
        & -x + 2y + z^2 \leq 0, \\
        & x, y, z \in \R.
\end{aligned}
\end{equation}
Las condiciones KKT de el nuevo problema so idénticas a las del problema original excepto que debe cumplir
\begin{equation*}
    v_1 (-x + 2y + z^2) = 0,
\end{equation*}
donde mantenemos la natación con $v_1$ por conveniencia.

Tal y como hemos visto anteriormente
\begin{equation*}
    u_1 = \frac{5}{3}, \quad
    v_1 = \frac{4}{3}, \quad
    x = \frac{185}{768}, \quad
    y = \frac{55}{768}, \quad
    z = -\frac{5}{16}, \quad
\end{equation*}
cumplen las condiciones KKT para el problema $(\ref{new_problem_c})$,
en este caso la función $h_1$ tiene como matriz Hessiana
\begin{equation*}
    \begin{pmatrix}
        0 & 0 & 0 \\
        0 & 0 & 0 \\
        0 & 0 & 2 \\
    \end{pmatrix},
\end{equation*}
que es semi-definida positiva y por tanto $h_1$ es convexa.
En este caso por tanto podemos aplicar el Teorema 5.17 para confirmar que $(\frac{185}{768}, \frac{55}{768}, -\frac{5}{16})$ es una solución del problema $(\ref{new_problem_c})$.
Como en el problema $(\ref{new_problem_c})$ la función $h_1$ esta activa,
se cumple que $-x + 2y + z^2 = 0$ y podemos afirmar que $(\frac{185}{768}, \frac{55}{768}, -\frac{5}{16})$ también es solución del problema original $(\ref{orginal_problem_c})$.

