\section*{3}

\begin{definition}
    Sea $S$ un conjunto convexo no vacío de un espacio vectorial real y $f : S \rightarrow \R$.
    \begin{enumerate}[label=\textup{(\alph*)}]
        \item Pseudoconvexidad estricta.
            \textup{
                Supóngase que $f$ tiene derivada direccional en un punto $\bar{x} \in S$ en cada dirección $x - \bar{x}$,
                con $x \in S$.
                Se dice que f es estrictamente pseudoconvexo en $\bar{x}$ si
                \begin{equation*}
                    f'(\bar{x})(x - \bar{x}) \geq 0, \text{ para } x \in S, \; x \neq \bar{x} \Rightarrow f(x) > f(\bar{x}).
                \end{equation*}
            }
            
            \item Cuasi convexidad fuerte.
                \textup{
                    Se dice que $f$ es cuasi convexo fuerte en $S$ si para cada $x_1 , x_2 \in S$, con $x_1 \neq x_2$,
                    se tiene que
                    \begin{equation*}
                        f(\lambda x_1 + (1 - \lambda) x_2) < \max \{ f(x_1), f(x_2) \}, \text{ para cada } \lambda \in (0, 1).
                    \end{equation*}
                }
    \end{enumerate}
\end{definition}
Se pide probar el siguiente resultado:

\begin{theorem}
    Sea $S$ un conjunto convexo no vacío de un espacio vectorial normado.
    \begin{enumerate}[label=\textup{(\alph*)}]
        \item 
            \textup{
                Considérese el problema
                \begin{equation*}
                    \min_{x \in S} f(x),
                \end{equation*}
                con $f : S \rightarrow \R$.
                Supóngase que $f$ es cuasi convexo fuerte en $S$.
                Si $\hat{x}$ es un mínimo local del problema,
                entonces $\hat{x}$ es la única solución global del problema.
            }
    
        \item 
            \textup{
                Sea $f$ un funcional definido en un conjunto abierto que contiene a $S$.
                Si $f$ es diferenciable Fréchet en cada punto $\hat{x} \in S$ y estrictamente pseudoconvexo en cada punto $\hat{x} \in S$,
                entonces $f$ es cuasi convexo fuerte en $S$.
            }
    \end{enumerate}
\end{theorem}

\noindent\rule{10cm}{0.4pt}

\subsection*{(a)}

Este aparatado es muy similar al problema 2 de la entrega del bloque 1.

Sea $x_0 \in S$ un mínimo local de un funcional cuasi convexo fuerte $f$.
Entonces $\exists \varepsilon > 0$ tal que $f(x_0) \leq f(x)$ para todo $x \in S \cap B(x_0, \varepsilon)$.

Consideremos un punto $x \in S \setminus B(x_0, \varepsilon)$, tal que $f(x) \neq f(x_0)$.
Definimos $\lambda := \frac{\varepsilon}{\| x_0 - x \|} \in (0, 1)$,
obteniendo $x_\lambda := \lambda x + (1 - \lambda) x_0 \in S$,
\begin{equation*}
    \| x_\lambda - x_0 \| = \| \lambda x + (1 - \lambda) x_0 - x_0 \| = \lambda \| x - x_0 \| = \varepsilon,
\end{equation*}
es decir $x_\lambda \in B(x_0, \varepsilon)$.

Por tanto, como $x_0 \neq x_\lambda$, utilizando la cuasi convexidad fuerte de $f$,
\begin{equation*}
    f(x_0) \leq f(x_\lambda) = f(\lambda x + (1 - \lambda) x_0 - x_0) < \max \{ f(x), f(x_0) \},
\end{equation*}
lo que implica $f(x_0) < f(x)$, por tanto para todo $x \in S$ tenemos que $f(x_0) = f(x)$ o bien $f(x_0) < f(x)$,
asi que $x_0$ es un mínimo global.


\subsection*{(b)}

Para demostrar la segunda parte del teorema seguiremos la idea de la demostración del teorema 4.18 del texto base.

Dados $x, y \in S$ tales que $x \neq y$,
supongamos que existe $\hat{\lambda} \in (0, 1)$ tal que
\begin{equation*}
    f(\hat{\lambda} x + (1 - \hat{\lambda}) y) \geq \max \{ f(x), f(y) \}.
\end{equation*}
Como $f$ es diferenciable Fréchet,
por el teorema 3.15 del texto base $f$ es continua,
por tanto existe $\bar{\lambda} \in (0, 1)$ tal que
\begin{equation*}
    f(\bar{\lambda} x + (1 - \bar{\lambda}) y) \geq f(\lambda x + (1 - \lambda) y), \text{ para todo } \lambda \in (0, 1).
\end{equation*}

Usando el teorema 3.13 y el teorema 3.8 (a) del texto base tenemos que para $\bar{x} := \bar{\lambda} x + (1 - \bar{\lambda}) y$
\begin{equation*}
    f'(\bar{x})(x - \bar{x}) \leq 0,
\end{equation*}
y
\begin{equation*}
    f'(\bar{x})(y - \bar{x}) \leq 0.
\end{equation*}
Con
\begin{equation}\label{ex3_b_ineq}
\begin{aligned}
    x - \bar{x} & = x - \bar{\lambda} x - (1 - \bar{\lambda}) y = (1 - \bar{\lambda}) (x - y), \\
    y - \bar{x} & = y - \bar{\lambda} x - (1 - \bar{\lambda}) y = - \bar{\lambda} (x - y), \\
\end{aligned}
\end{equation}
usando la linearidad de $f'(\bar{x})$ obtenemos
\begin{equation*}
    0 \geq f'(\bar{x})(x - \bar{x}) = (1 - \bar{\lambda}) f'(\bar{x}) (x - y),
\end{equation*}
y
\begin{equation*}
    0 \geq f'(\bar{x})(y - \bar{x}) = - \bar{\lambda} f'(\bar{x}) (x - y).
\end{equation*}
Por tanto tenemos que $f'(\bar{x})(x - y) = 0$,
y usando la igualdad $(\ref{ex3_b_ineq})$ también tenemos $f'(\bar{x})(y - \bar{x}) = 0$.

Como hemos asumido que $f$ es estrictamente pseudoconvexo en cada punto $x \in S$,
$f$ es estrictamente pseudoconvexo en $\bar{x}$ y por tanto
\begin{equation*}
    f(y) - f(\bar{x}) > 0.
\end{equation*}

Pero esta desigualad contradice la siguiente desigualad
\begin{equation*}
\begin{aligned}
    f(y) - f(\bar{x})
        & = f(y) - f(\bar{\lambda} x + (1 - \bar{\lambda}) y) \\
        & \leq f(y) - f(\hat{\lambda} x + (1 - \hat{\lambda}) y) \\
        & \leq f(y) - \max \{ f(x), f(y) \} \\
        & \leq 0. \\
\end{aligned}
\end{equation*}

Por tanto para todo $\lambda \in (0,1)$
\begin{equation*}
    f(\lambda x + (1 - \lambda) y) < \max \{ f(x), f(y) \},
\end{equation*}
y consecuentemente $f$ es cuasi convexo fuerte.
