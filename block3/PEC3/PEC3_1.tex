\section*{1}

\begin{definition}
    Sea $(X, \| \cdot \|)$ un espacio vectorial normado y $D \subset X$ un cono convexo,
    con $X \neq D \neq \{ 0_X \}$.
    \begin{enumerate}[label=\textup{(\alph*)}]
        \item Base de un cono.
            \textup{
                Se dice que un conjunto no vacío y convexo $B \subset D$ es una base de $D$,
                si cada elemento $x \in D \setminus \{ 0_X \}$ admite una única representación de la forma
                \begin{equation*}
                    x = \lambda b, \text{ con } \lambda > 0 \text{ y } b \in B.
                \end{equation*}
            }

        \item Cono polar.
            \textup{
                El cono
                \begin{equation*}
                    D^* := \{ \mu \in X^* : \mu(d) \geq 0, \; \forall d \in D \},
                \end{equation*}
                recibe el nombre de cono polar o cono dual de $D$
                ($X^*$ denota el espacio dual topológico de $X$).
            }
        \item Cono polar estricto.
            \textup{
                El conjunto 
                \begin{equation*}
                    D^{s*} := \{ \mu \in X^* : \mu(d) > 0, \; \forall d \in D \setminus \{ 0_X \} \},
                \end{equation*}
                recibe el nombre de cono polar estricto de $D$.
            }
    \end{enumerate}
\end{definition}

Para los conceptos introducidos en la definición anterior,
se pide probar las propiedades que se indican a continuación.
En lo que sigue $(X, \| \cdot \|)$ es un espacio vectorial normado y $D \subset X$ un cono convexo, con $X \neq D \neq \{ 0_X \}$.

\noindent \textit{Propiedades.}
\begin{enumerate}
    \item Supóngase que int $D \neq \emptyset$. Se tiene que
        \begin{equation*}
            \text{int} D \subset \{ x \in X : \mu(x) > 0, \forall \mu \in D^{*} \setminus \{ 0_{X^*} \} \}.
        \end{equation*}

    \item Si int $D \neq \emptyset$, entonces $D^{*}$ es puntiagudo (pointed).
    \item Si $D$ tiene una base, entonces $D$ es puntiagudo (pointed).
    \item Supóngase que $D^{s*} \neq \emptyset$.
        Entonces, para cada $\mu \in D^{s*}$ se tiene que el conjunto
        \begin{equation*}
            B := \{ d \in D : \mu(d) = 1 \},
        \end{equation*}
        es una base de $D$.
    \item Si $D^{s*} \neq \emptyset$, entonces $D$ es puntiagudo.
\end{enumerate}

\noindent\rule{10cm}{0.4pt}

\subsection*{1}

Para todo $x \in \text{int}D$ es obvio que $x \in X$,
ademas para todo $\mu \in D^* \setminus \{ 0_{X^*} \}$, tenemos que $\mu(x) \geq 0$.

Sea $x \in \text{int}D$ supongamos que existe $\mu \in D^* \setminus \{ 0_{X^*} \}$ tal que $\mu(x) = 0$.
Entonces como $x \in \text{int}D \neq \emptyset$ tenemos que existe $r > 0$ tal que la bola abierta $B(x, r) \subset \text{int} D$.

Tenemos que existe $w \in \text{int}D \cap B(0, r)$ tal que $w > 0$,
si no existiera tendiéramos que para todo $v \in \text{int}D \cap B(0, r) \subset D$, $\mu(v) = 0$,
por tanto $\mu = 0_{X^*}$, contradiciendo la hipótesis.
Por la linealidad de $\mu$ tenemos que $y = x - w \in B(x, r) \subset \text{int} D \subset D$ y
\begin{equation*}
    \mu(y) = \mu(x) - \mu(w) = - \mu(w) < 0,
\end{equation*} 
contradiciendo que $\mu \in D^*$!!

Por tanto para todo $\mu \in D^* \setminus \{ 0_{X^*} \}$ se tiene que $\mu(x) > 0$,
y esto implica que 
\begin{equation*}
    \text{int} D \subset \{ x \in X : \mu(x) > 0, \forall \mu \in D^{*} \setminus \{ 0_{X^*} \} \}.
\end{equation*}

Notemos que en este caso $D \neq X$ ya que si $D = X$ el único elemento del dual $D^*$ seria $0_{X^*}$,
dado que para cualquier otro funcional $\mu$ si existe $d \in D$ tal que $\mu(d) > 0$,
como $D = X$ tenemos que $-d \in D$ y por tanto $\mu(-d) = - \mu(d) < 0$ contradiciendo que es del dual.

\subsection*{2}

Por el aparatado anterior,
como $\text{int} D \neq \emptyset$ para todo $x \in \text{int}D$ y $\mu \in D^* \setminus \{ 0_{X^*} \}$ tenemos $\mu(x) > 0$.

Si $-\mu \in D^*$ tenemos que para todo $x \in \text{int}D$, $-\mu(x) > 0$,
que implica $\mu(x) < 0$, contradiciendo que $\mu(x) > 0$!!

Resultando en que si $\mu \in D^*$ y $-\mu \in D^*$ entonces $\mu = 0_{X^*}$,
y por tanto $D^*$ es puntiagudo.

\subsection*{3}

Sea $x \in D \setminus \{ 0_X \}$ y $B$ una base de $D$.
Entonces existe un único $b \in B$ y $\lambda > 0$ tal que $x = \lambda b$.

Si $-x \in D \setminus \{ 0_X \}$,
entonces, como $B$ es convexo,
para todo $\alpha \in [0, 1]$ tenemos que
\begin{equation*}
    b_\alpha = \alpha b - (1 - \alpha) b = (2 \alpha - 1) b \in B.
\end{equation*}
Y $x = \lambda b = 2 \lambda b_{0.75}$,
contradiciendo que la representación de $x = \lambda b$ con $\lambda > 0$ y $b \in B$ es única.
Por tanto $x = 0_X$ y consecuentemente $D$ es puntiagudo.

\subsection*{4}

Dado $\mu \in D^{s*}$,
veamos que para cada $x \in D \setminus \{ 0_X \}$ existe un único $b \in B$ y $\lambda > 0$.

Supongamos que existen $b_1, b_2 \in B$ y $\lambda_1, \lambda_2 > 0$,
tales que $x = \lambda_1 b_1 = \lambda_2 b_2$.
Tenemos que
\begin{equation*}
    \mu(x) = \lambda_1 = \lambda_1 \mu(b_1) = \lambda_2 \mu(b_2) = \lambda_2,
\end{equation*}
por tanto $\lambda_1 = \lambda_2$, y como consecuencia
\begin{equation*}
    x = \lambda_1 b_1 = \lambda_1 b_2,
\end{equation*}
Resultando en $b_1 = b_2$.

Veamos ahora que dado $\mu \in D^{s*}$, $B$ genera,
i.e. para todo $x \in D \setminus \{ 0_X \}$ existe $b \in B$ y $\lambda > 0$ tal que $x = \lambda b$.

Supongamos que existe $x \in D \setminus \{ 0_X \}$ tal que para todo $b \in B$ y para todo $\lambda > 0$,
$x \neq \lambda b$.

Como $D$ es un cono se tiene que para todo $\alpha > 0$, $\alpha x \in D$.
% Si $\alpha x = \lambda b$, por algún $b \in B$ y $\lambda > 0$,
% entonces
% \begin{equation*}
%     x = \frac{\lambda}{\alpha} b,
% \end{equation*}
% contradiciendo la hipótesis que no existe $b \in B$ y $\lambda > 0$ tal que $x = \lambda b$.
% Por tanto $\alpha x \neq \lambda b$ para todo $b \in B$ y $\lambda > 0$.
Entonces
\begin{equation*}
    \mu(\alpha x) = \alpha \mu(x),
\end{equation*}
usando $\alpha = \frac{1}{\mu(x)}$,
tenemos que $\mu(\alpha x) = 1$ y por tanto $\alpha x \in B$,
contradiciendo la hipótesis inicial.

\subsection*{5}

Usando el resultado anterior, como $D^{s*} \neq \emptyset$,
tenemos que $D$ tiene una base,
por tanto usando la propiedad 3 tenemos que $D$ es puntiagudo.
