\section*{4}

Resuelva el ejercicio (2.8) de la página 31 del texto base.

\noindent\rule{10cm}{0.4pt}

Inspirándonos en la demostración del Teorema 2.23 del texto base,
debemos ver que el nuevo funcional $f$ es quasi convexo y continuo,
ya que el conjunto $S$ no ha cambiado y sigue siendo cerrado, acotado y convexo.
De esta manera resolveríamos el ejercicio aplicando el Teorema 2.12 del texto base.

Veamos que el funcional $f$ es convexo.

Dados $u, v \in S$ y $\lambda \in [0,1]$,
\begin{equation*}
\begin{aligned}
    f(\lambda u + (1 - \lambda) v)
        & = \max_{t \in [t_0, t_1]} \left| x_0 - \hat{x}(t) + \int^{t}_{t_0} e^{A (t - s)} B (\lambda u(s) + (1 - \lambda) v(s)) \right| \\
        & = \max_{t \in [t_0, t_1]} \left| x_0 - \hat{x}(t) + \lambda \int^{t}_{t_0} e^{A (t - s)} B u(s) + (1 - \lambda) \int^{t}_{t_0} e^{A (t - s)} B v(s) \right| \\
        & = \max_{t \in [t_0, t_1]} \left| \lambda \left( x_0 - \hat{x}(t) \int^{t}_{t_0} e^{A (t - s)} B u(s) \right) 
            + (1 - \lambda) \left( x_0 - \hat{x}(t) \int^{t}_{t_0} e^{A (t - s)} B v(s) \right) \right| \\
        & \leq \max_{t \in [t_0, t_1]} \left( \lambda \left| x_0 - \hat{x}(t) \int^{t}_{t_0} e^{A (t - s)} B u(s) \right| 
            + (1 - \lambda) \left| x_0 - \hat{x}(t) \int^{t}_{t_0} e^{A (t - s)} B v(s) \right| \right) \\
        & \leq \lambda \max_{t \in [t_0, t_1]} \left| x_0 - \hat{x}(t) \int^{t}_{t_0} e^{A (t - s)} B u(s) \right| 
            + (1 - \lambda) \max_{t \in [t_0, t_1]} \left| x_0 - \hat{x}(t) \int^{t}_{t_0} e^{A (t - s)} B v(s) \right| \\
        & = \lambda f(u) + (1 - \lambda) f(v),
\end{aligned}   
\end{equation*}
donde hemos usado la desigualdad triangular y el hecho que el máximo de la suma de dos funciones positivas es menor que la suma de sus máximos.

Para ver la continuidad de $f$,
definimos el operador lineal
\begin{equation*}
    Lu(t) = \int^{t}_{t_0} e^{A (t - s)} B u(s) ds,
\end{equation*}
que es acotado y por tanto continuo,
como se muestra en la demostración del Teorema 2.23 del texto base y en un documento proporcionado por la coordinadora de la asignatura.

El valor absoluto es una función continua, por tanto aplicado a otra función continua el resultado sigue siendo una función continua,
con esto tenemos que $| x_0 - \hat{x}(t) + Lu(t) |$ es continuo.

Finalmente como $L$ es continuo en un conjunto compacto es también uniformemente continuo con respecto a $t$,
por tanto el máximo respecto a $t$ de la function $| x_0 - \hat{x}(t) + Lu(t) |$ es continuo,
resultando en que $f$ es continuo.


