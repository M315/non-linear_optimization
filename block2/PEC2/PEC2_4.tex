\section*{4}

Sea $(X, \| \cdot \|)$ un espacio real normado y $f : X \rightarrow \R := \R \cup \{-\infty, +\infty\}$.
La función $f^* : X^* \rightarrow \R$ definida por
\begin{equation*}
    f^* (x^*) = \sup_{x \in X} \{ \langle x, x^* \rangle - f(x) \},
\end{equation*}
se denomina la conjugada de Fenchel de $f$ ($X^*$ denota el conjunto de funcionales lineales y continuos $l : X \rightarrow \R$).

\begin{enumerate}[label=(\alph*)]
    \item Pruebe que
        \begin{equation*}
            f(x) + f^* (x^*) \geq \langle x, x^* \rangle, \quad \forall x \in X, \quad x^* \in X^*.
        \end{equation*}
        La desigualdad anterior se conoce como \textit{desigualdad de Young-Fenchel}.
    
    \item Sea $x_0 \in \text{dom} f$. Demuestre que
        \begin{equation*}
            x^* \in \partial f(x_0)
        \end{equation*}
        si y sólo si
        \begin{equation}\label{ex4_eq1}
            f(x) + f^*(x^*) = \langle x_0, x^* \rangle,
        \end{equation}
        y que lo anterior implica que $x_0 \in \partial f^*(x^*)$.
    
    \item Sea $x_0 \in \text{dom} f$. Demuestre que
        \begin{equation*}
            \partial f(x_0) \neq \emptyset 
                \iff f(x_0) = \max_{x^* \in X^*} (\langle x_0, x^* \rangle - f^*(x^*)).
        \end{equation*}
    
\end{enumerate}

\noindent\rule{10cm}{0.4pt}

\subsection*{(a)}

Para cualquier $x \in X$ y $x^* \in X^*$, tenemos
\begin{equation*}
\begin{aligned}
    f(x) + f^*(x^*)
        & = f(x) + \sup_{\hat{x} \in X} \{ \langle \hat{x}, x^* \rangle -f(\hat{x}) \} \\
        & \geq f(x) + \langle x, x^* \rangle - f(x) \\
        & = \langle x, x^* \rangle.
\end{aligned}
\end{equation*}
Como queríamos ver.


\subsection*{(b)}

Supongamos que $x^* \in \partial f(x_0)$,
entonces,
\begin{equation*}
\begin{aligned}
    f(x) & \geq f(x_0) + \langle x - x_0, x^* \rangle, \; \forall x \in X, \\
    & \Rightarrow \langle x_0, x^* \rangle \geq f(x_0) + \{ \langle x, x^* \rangle - f(x) \} , \; \forall x \in X, \\
    & \Rightarrow \langle x_0, x^* \rangle \geq f(x_0) + f^*(x^*). \\
\end{aligned} 
\end{equation*}
Usando el anterior resultado junto al resultado del apartado anterior,
tenemos que
\begin{equation*}
    \langle x_0, x^* \rangle = f(x_0) + f^*(x^*).
\end{equation*}

Supongamos ahora que $x^*$ cumple $(\ref{ex4_eq1})$,
entonces para todo $x \in X$, tenemos
\begin{equation*}
\begin{aligned}
    f(x_0) + \langle x - x_0, x^* \rangle
        & = \langle x, x^* \rangle - \langle x_0, x^* \rangle + \langle x_0, x^* \rangle - f^*(x^*) \\
        & = \langle x, x^* \rangle - \sup_{\hat{x} \in X} \{ \langle \hat{x}, x^* \rangle -f(\hat{x}) \} \\
        & \leq \langle x, x^* \rangle - (\langle x, x^* \rangle - f(x))  \\
        & = f(x),  \\
\end{aligned}
\end{equation*}
por tanto $x^* \in \partial f(x_0)$.

Finalmente veamos que si $x^* \in X^*$ y $x_0 \in X$ cumplen $(\ref{ex4_eq1})$,
entonces se tiene que para cualquier $y^* \in X^*$
\begin{equation*}
\begin{aligned}
    f^*(x^*) + \langle x_0, y^* - x^* \rangle
        & = - f(x_0) + \langle x_0, x^* \rangle - \langle x_0, x^* \rangle + \langle x_0, y^* \rangle \\
        & = \langle x_0, y^* \rangle - f(x_0) \\
        & \leq \sup_{x \in X} \{ \langle x, y^* \rangle - f(x) \} \\
        & = f^*(y^*), \\
\end{aligned}
\end{equation*}
por tanto $x_0 \in \partial f^*(x^*)$.


\subsection*{(c)}

Supongamos que $\partial f(x_0) \neq \emptyset$ y que $y^* \in \partial f(x_0)$,
por la desigualdad de Young-Fenchel tenemos que
\begin{equation*}
    f(x_0) \geq \langle x_0, x^* \rangle - f^*(x^*), \; \forall x^* \in X^*,
\end{equation*}
lo que implica que
\begin{equation}\label{ex4_c_eq}
\begin{aligned}
    f(x_0) 
        & \geq \max_{x^* \in X^*} (\langle x_0, x^* \rangle - f^*(x^*)) \\
        & \geq \langle x_0, y^* \rangle - f^*(y^*) \\
        & = f(x_0), \\
\end{aligned}
\end{equation}
donde en la ultima igualdad hemos usado el resultado del apartado anterior.
Por tanto las desigualdades de $(\ref{ex4_c_eq})$ son igualdades y tenemos
\begin{equation*}
        f(x_0) = \max_{x^* \in X^*} (\langle x_0, x^* \rangle - f^*(x^*)).
\end{equation*}

Supongamos ahora que
\begin{equation*}
        f(x_0) = \max_{x^* \in X^*} (\langle x_0, x^* \rangle - f^*(x^*)).
\end{equation*}
Sea $y^* \in X^*$ tal que
\begin{equation*}
    \langle x_0, y^* \rangle - f^*(y^*) \geq \langle x_0, x^* \rangle - f^*(x^*),
\end{equation*}
es decir $y^*$ es un máximo de la función $\langle x_0, x^* \rangle - f^*(x^*)$.
Entonces tenemos que
\begin{equation*}
    f(x_0) + f^*(y^*) = \langle x_0, y^* \rangle,
\end{equation*}
por tanto usando el resultado del apartado anterior tenemos que $y^* \in \partial f(x_0)$,
y por tanto $\partial f(x_0) \neq \emptyset$.

