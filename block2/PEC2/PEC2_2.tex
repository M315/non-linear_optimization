\section*{2}

Sea $\| \cdot \|_{\diamond}$ una norma en $\R^n$ y sea
\begin{equation*}
    f(x) = \| x \|_{\diamond}.
\end{equation*}
Se define la norma dual del siguiente modo:
\begin{equation*}
    \| l \|_{*} 
        = \max_{h \neq 0} \frac{| \langle l, h \rangle |}{\| h \|_{\diamond}}
        = \max_{\| h \|_{\diamond} = 1} | \langle l, h \rangle |.
\end{equation*}
Pruebe que se cumplen las siguientes propiedades:
\begin{enumerate}[label=(\alph*)]
    \item $\partial f(0) = \{ l \in \R^n : \| l \|_{*} \leq 1 \}$. En particular, pruebe que
        \begin{enumerate}[label=(\alph{enumi}.\arabic*)]
            \item $\partial \| 0 \|_{2} = \{ l \in \R^n : \| l \|_{2} \leq 1 \}$,
            \item $\partial \| 0 \|_{1} = \{ l \in \R^n : \| l \|_{\infty} \leq 1 \}$,
            \item $\partial \| 0 \|_{\infty} = \{ l \in \R^n : \| l \|_{1} \leq 1 \}$,
        \end{enumerate}

    \item $\partial f(x) = \{ l \in \R^n : \| l \|_{*} \leq 1, \; \langle l, x \rangle = \| x \|_{\diamond} \}$,
        para cada $x \in \R^n$.
    
\end{enumerate}

\noindent\rule{10cm}{0.4pt}

\subsection*{(a)}

Aplicando la definición de subgradiente tenemos
\begin{equation*}
\begin{aligned}
    \partial \| 0 \|_{\diamond}
        & = \{ l \in \R^n : \| x \|_{\diamond} \geq l(x), \; \forall x \in \R^n \} \\
        & = \{ l \in \R^n : \frac{| l(x) |}{\| x \|_{\diamond}} \leq 1, \; \forall x \in \R^n \setminus \{ 0 \} \} \\
        & = \{ l \in \R^n : \| l \|_{*} \leq 1 \}. \\
\end{aligned}
\end{equation*}

Los siguientes subapartados son consecuencia del resultado anterior y que el dual de la norma $\ell^2$ es $\ell^2$,
el dual de $\ell^1$ es $\ell^\infty$ y el dual de $\ell^\infty$ es $\ell^1$.

\subsubsection*{Norma dual de $\ell^2$}

Para ver que la norma dual de $\ell^2$ es ella misma usamos la siguiente definición norma dual,
para cualquier $y \in (\R^n)^* = \R^n$,
\begin{equation*}
    \| y \|_{2}^{*} = \sup_{\| x \|_2 \leq 1} | \langle x, y \rangle |.
\end{equation*}
Usando la desigualdad de Cauchy-Schwarz tenemos,
\begin{equation*}
        | \langle x, y \rangle |
        \leq \| x \|_2 \| y \|_2
        \leq \| y \|_2,
\end{equation*},
donde en la ultima desigualdad usamos $\| x \| \leq 1$.
Finalmente si usamos $x = \frac{y}{\| y \|_2}$, que cumple $\| x \| = 1$,
tenemos
\begin{equation*}
    | \langle x, y \rangle | = \| y \|_2,
\end{equation*}
alcanzando el supremo y por tanto
\begin{equation*}
    \| y \|_{2}^{*} = \| y \|_2.
\end{equation*}

\subsubsection*{Norma dual de $\ell^\infty$}

Veamos que la norma dual de $\ell^\infty$ es $\ell^1$,
tenemos que para cualquier $y \in (\R^n)^* = \R^n$,
\begin{equation*}
    \| y \|_{\infty}^{*} = \sup_{\| x \|_\infty \leq 1} | \langle x, y \rangle |
        = \sup_{\max_{1 \leq i \leq n} |x_i| \leq 1} \left| \sum_{i = 1}^{n} x_i y_i \right|.
\end{equation*}
La restricción $\max_{1 \leq i \leq n} |x_i| \leq 1$,
implica que $x \in [-1, 1]^n \subset \R^n$.

Para maximizar la suma elegimos $x \in[-1, 1]^n$ tal que
\begin{equation*}
    x_i = \left\{
    \begin{aligned}
        1     & \text{ si } y_i \geq 0, \\
        -1    & \text{ si } y_i < 0.
    \end{aligned}
        \right.
\end{equation*}
De este modo se tiene
\begin{equation*}
    | \langle x, y \rangle | = \sum_{i = 1}^{n} |y_i|,
\end{equation*}
i por tanto
\begin{equation*}
    \| y \|_{\infty}^{*} = \sum_{i = 1}^{n} |y_i| = \| y \|_1.
\end{equation*}

\subsubsection*{Norma dual de $\ell^1$}

Finalmente veamos que la norma dual de $\ell^1$ es $\ell^\infty$,
tenemos que para cualquier $y \in (\R^n)^* = \R^n$,
\begin{equation*}
    \| y \|_{1}^{*} = \sup_{\| x \|_1 \leq 1} | \langle x, y \rangle |
        = \sup_{\sum_{i = 1}^{n} |x_i| \leq 1} \left| \sum_{i = 1}^{n} x_i y_i \right|.
\end{equation*}

Para maximizar $ | \langle x, y \rangle | $ con la restricción $\| x \|_1 \leq 1$,
los componente $|x_i|$ deberían estar concentrados en maximizar los valores mas grandes $|y_i$.
En particular el máximo se alcanza cuando $x_k = \text{signo}(y_k)$ para $|y_k|$ el componente mas grande de $y$ y $x_i = 0$ para todo $i \neq k$.
Esto es, sea $k = \text{argmax}_{1 \leq i \leq n} |y_i|$, entonces
\begin{equation*}
    x_i = \left\{
    \begin{aligned}
        \text{signo}(y_k)   & \text{ si } i = k, \\
        0                   & \text{ si } i \neq k.
    \end{aligned}
        \right.
\end{equation*}
En tal caso la norma dual es
\begin{equation*}
    \| y \|_{1}^{*} = \max_{1 \leq i \leq n} |y_i| = \| y \|_\infty.
\end{equation*}



\subsection*{(b)}

Veamos que 
\begin{equation*}
    \partial f(x) = \{ l \in \R^n : \| l \|_{*} \leq 1, \; \langle l, x \rangle = \| x \|_{\diamond} \}, \; \forall x \in \R^n,
\end{equation*}
en el caso $x = 0$, ya hemos visto en el apartado anterior que se cumple,
por tanto veamos que se cumple para $x \neq 0$.


Sea $l \in \R^n$ tal que $\| l \|_{*} \leq 1$ y $\langle l, x \rangle = \| x \|_{\diamond}$,
tenemos que para cualquier $y \in \R^n$,
\begin{equation*}
\begin{aligned}
    \| x \|_{\diamond} + \langle l, y - x \rangle
        & = \| x \|_{\diamond} + \langle l, y \rangle - \langle l, x \rangle \\
        & = \| x \|_{\diamond} + \langle l, y \rangle - \| x \|_{\diamond} \\
        & = \langle l, y \rangle \\ 
        & \leq \| y \|_{\diamond} \cdot \frac{|\langle l, y \rangle|}{\| y \|_{\diamond}} \\
        & \leq \| y \|_{\diamond} \| l \|_{*} \\
        & \leq \| y \|_{\diamond},
\end{aligned}
\end{equation*}
por tanto $l \in \partial \| x \|_{\diamond}$,
que implica $\{ l \in \R^n : \| l \|_{*} \leq 1, \; \langle l, x \rangle = \| x \|_{\diamond} \} \subset \partial f(x)$,
para todo $x \in \R^n$.

Sea $l \in \partial f(x)$,
tenemos que
\begin{equation*}
    \| x \|_{\diamond} - \langle l, x \rangle 
        = \| 2x \|_{\diamond} - \| x \|_{\diamond} - \langle l, 2x - x \rangle
        \geq 0,
\end{equation*}
y
\begin{equation*}
    - \| x \|_{\diamond} + \langle l, x \rangle 
        = \| 0 \|_{\diamond} - \| x \|_{\diamond} - \langle l, 0 - x \rangle
        \geq 0.
\end{equation*}
Las dos desigualdades llevan a $\| x \|_{\diamond} = \langle l, x \rangle$.
Ademas para todo $y \in \R^n$, se cumple
\begin{equation*}
\begin{aligned}
    \| y \|_{\diamond}
        & \geq \| x \|_{\diamond} + \langle l, y - x \rangle \\
        & = \| x \|_{\diamond} + \langle l, y \rangle - \| x \|_{\diamond} \\
        & = \langle l, y \rangle.
\end{aligned}
\end{equation*}
Por lo tanto podemos concluir que
\begin{equation*}
    \| l \|_{*} 
        = \sup_{y \in \R^n} \frac{|\langle l, y \rangle}{\| y \|_{\diamond}} 
        \leq 1.
\end{equation*}
Por tanto se cumple $\partial f(x) \subset \{ l \in \R^n : \| l \|_{*} \leq 1, \; \langle l, x \rangle = \| x \|_{\diamond} \}$, para todo $x \in \R^n$.
Y con ambas inclusiones concluimos que $\partial f(x) = \{ l \in \R^n : \| l \|_{*} \leq 1, \; \langle l, x \rangle = \| x \|_{\diamond} \}$.